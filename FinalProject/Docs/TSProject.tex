\documentclass[conference]{IEEEtran}
\IEEEoverridecommandlockouts
% The preceding line is only needed to identify funding in the first footnote. If that is unneeded, please comment it out.
\usepackage{cite}
\usepackage{amsmath,amssymb,amsfonts}
\usepackage{algorithmic}
\usepackage{graphicx}
\usepackage{textcomp}
\usepackage{xcolor}
\def\BibTeX{{\rm B\kern-.05em{\sc i\kern-.025em b}\kern-.08em
		T\kern-.1667em\lower.7ex\hbox{E}\kern-.125emX}}
\begin{document}
	
	\title{Project Management Chatbot}
	
	\author{\IEEEauthorblockN{ Nahin Peñaranda}
		\IEEEauthorblockA{20231020032\\
			\textit{Systems Engineering } \\
			njpenarandam@udistrital.edu.co}
		\and
		\IEEEauthorblockN{Manuel Guerrero}
		\IEEEauthorblockA{20231020078\\
			\textit{Systems Engineering } \\
			mrguerreroc@udistrital.edu.co}
		
	}
	
	\maketitle
	
	\begin{abstract}
		This document describes the system analysis of a Project management Chatbot, from Systems thinking and his properties, and give a help toot to project management.
	\end{abstract}
	
	\begin{IEEEkeywords}
		system, analysis, project, stakeholders
	\end{IEEEkeywords}
	
	\section{Introduction}
	Project management consists, basically, in the organization of a project, where a client requires objectives to be completed, so this have a process with planning, organizing, leading, controlling resources to complete the goals at a time frame, this involves coordinating tasks, managing people, overseeing activities, etc in a efficiently and effectively way.
	
	\section{Why a Chatbot?}
	
	A chatbot is program which simulates an human conversation with a final user, in the new times, those chatbots are equipped with artificial intelligence, which offers a better conversation, to answer questions, solve problems, give recommendations, etc. In our context, a chatbot allow to offer up a help in project management in all parts of a project.
	
	With external information, our chatbot can be entrenaited to give tips to better planning, organizing, leading, controlling resources, making more efficiently and effectively the solution.
	
	\section{Method and  materials}
	Before you begin to format your paper, first write and save the content as a 
	separate text file. Complete all content and organizational editing before 
	formatting. Please note sections \ref{AA}--\ref{SCM} below for more information on 
	proofreading, spelling and grammar.
	
	Keep your text and graphic files separate until after the text has been 
	formatted and styled. Do not number text heads---{\LaTeX} will do that 
	for you.
	
	
	\begin{thebibliography}{00}
		\bibitem{b1} G. Eason, B. Noble, and I. N. Sneddon, ``On certain integrals of Lipschitz-Hankel type involving products of Bessel functions,'' Phil. Trans. Roy. Soc. London, vol. A247, pp. 529--551, April 1955.
		\bibitem{b2} J. Clerk Maxwell, A Treatise on Electricity and Magnetism, 3rd ed., vol. 2. Oxford: Clarendon, 1892, pp.68--73.
		\bibitem{b3} I. S. Jacobs and C. P. Bean, ``Fine particles, thin films and exchange anisotropy,'' in Magnetism, vol. III, G. T. Rado and H. Suhl, Eds. New York: Academic, 1963, pp. 271--350.
		\bibitem{b4} K. Elissa, ``Title of paper if known,'' unpublished.
		\bibitem{b5} R. Nicole, ``Title of paper with only first word capitalized,'' J. Name Stand. Abbrev., in press.
		\bibitem{b6} Y. Yorozu, M. Hirano, K. Oka, and Y. Tagawa, ``Electron spectroscopy studies on magneto-optical media and plastic substrate interface,'' IEEE Transl. J. Magn. Japan, vol. 2, pp. 740--741, August 1987 [Digests 9th Annual Conf. Magnetics Japan, p. 301, 1982].
		\bibitem{b7} M. Young, The Technical Writer's Handbook. Mill Valley, CA: University Science, 1989.
	\end{thebibliography}
	\vspace{12pt}
	\color{red}
	IEEE conference templates contain guidance text for composing and formatting conference papers. Please ensure that all template text is removed from your conference paper prior to submission to the conference. Failure to remove the template text from your paper may result in your paper not being published.
	
\end{document}