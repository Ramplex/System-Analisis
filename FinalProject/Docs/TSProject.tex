\documentclass[conference]{IEEEtran}
\IEEEoverridecommandlockouts
% The preceding line is only needed to identify funding in the first footnote. If that is unneeded, please comment it out.
\usepackage{cite}
\usepackage{amsmath,amssymb,amsfonts}
\usepackage{algorithmic}
\usepackage{graphicx}
\usepackage{textcomp}
\usepackage{xcolor}
\def\BibTeX{{\rm B\kern-.05em{\sc i\kern-.025em b}\kern-.08em
		T\kern-.1667em\lower.7ex\hbox{E}\kern-.125emX}}
\begin{document}
	
	\title{Project Management Chatbot}
	
	\author{\IEEEauthorblockN{ Nahin Peñaranda}
		\IEEEauthorblockA{20231020032\\
			\textit{Systems Engineering } \\
			njpenarandam@udistrital.edu.co}
		\and
		\IEEEauthorblockN{Manuel Guerrero}
		\IEEEauthorblockA{20231020078\\
			\textit{Systems Engineering } \\
			mrguerreroc@udistrital.edu.co}
		
	}
	
	\maketitle
	
	\begin{abstract}
		This document describes the system analysis of a Project management Chatbot, from Systems thinking and his properties, and give a help toot to project management.
	\end{abstract}
	
	\begin{IEEEkeywords}
		system, analysis, project, stakeholders
	\end{IEEEkeywords}
	
	\section{Introduction}
	Project management consists, basically, in the organization of a project, where a client requires objectives to be completed, so this have a process with planning, organizing, leading, controlling resources to complete the goals at a time frame, this involves coordinating tasks, managing people, overseeing activities, etc in a efficiently and effectively way.
	
	\section{Why a Chatbot?}
	
	A chatbot is program which simulates an human conversation with a final user, in the new times, those chatbots are equipped with artificial intelligence, which offers a better conversation, to answer questions, solve problems, give recommendations, etc. In our context, a chatbot allow to offer up a help in project management in all parts of a project.
	
	With external information, our chatbot can be entrained to give tips for better planning, organizing, leading, controlling resources, making more efficiently and effectively the solution.
	
	\section{System Analisys}
	To understand the construction of a chatbot, we need to understand the behavior from of system point of view.\\
	
	1. Split parts is basic for an application development (\textbf{Holistic approach}), so we need layers to achieve our goal, we need, frontend, the layer for user interface, backend, all the logic with the brain of chatbot (LlaMa 3), API layer, basically, communication between backend and frontend, and a storage for persistence. \\
	
	Relations between all those parts can see the synergy of application.\\
	
	2. Chaos theory is important to get a better behavior of app, in a perfect world, our chatbot will have perfect performance, but in real world, we have to consider some things:\\
	
		A. Many users, at same time, is going to cause overload, equals, slow responses.\\
		
		B. Hardware limitation: Low capacity to handle user, can be a problem for the app.\\
		
		C. IA: Is a powerful tool, but, is artificial, that can causes mistakes, inaccuracies, etc, so feedback is useful to improve better sensibility. \\
		
	
	3. 
	
	
	\begin{thebibliography}{00}
		\bibitem{b1} 
	\end{thebibliography}
	
	
\end{document}